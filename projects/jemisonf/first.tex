\documentclass[12pt]{article}
%\usepackage{times}%
%this is a comment
\title{Vision Statement}
\author{Fischer Jemison and Sean Gillen}




\begin{document}
\maketitle
\tableofcontents


\section{Problem Statement}
not being able to view rendered markdown as you’re working on it. In the past I’ve uploaded stuff to github in order to check what the markdown I’m writing looks like.
\section{Solution}
A c++ program that can convert markdown to pdf and html for easy viewing. This will provide a command line interface for doing conversions as well as a set of libraries to allow users to create their own programs for parsing markdown.
\section{Advantages of our Solution}
\begin{itemize}
	\item Speed: c++ is difficult to program in, so it is not the easiest tool to use for this problem, but will outperform tools made in other languages
	\item Modularization: the program will be broken into components that others can easily use
	\item Open Source: make it easier for others to use our code while lengthening the lifespan of the projecting by making it so that the original developers aren't solely responsible for maintaining the code.
\end{itemize}
\section{Technology Used}
\begin{itemize}
	\item c++ 
	\item pdf
	\item html/css
\end{itemize}
%\section{Plain Text}
%Hello, world!
%\subsection{Bold Text}
%{\bf Hello, world!}
%\subsubsection{Bold and Large Text}
%{\Large \bf Hello, world!!!}
%
%\section{Textbook}
%There is no required textbook. Here are some useful books on software engineering
%
%Software Engineering: Theory and Practice~\cite{pfleeger2010software}
%
%Software Engineering~\cite{sommerville2011software}
%
%
%\bibliography{myref}
%\bibliographystyle{plain}

\end{document}
