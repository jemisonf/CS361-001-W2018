\section{Design Changes}
One of the things that came up was how to store the text data stuff. The text is going to come in through a file that is provided by the user. Since the file is in markdown, all the formatting with the text also needs to be streamed in. So, the question that followed was how to store this information. A couple methods could have been utilized such as through a linked list vector, or a queue. However, both of these accomplishes the same task, it was choosing which one would be more effective. \newline 

\noindent The customers had guided the group into choosing a queue to store the information that would be streamed in from the document. So, it would have the structure of first in and first out with the text. This would then be stored in a class that be passed between the different components. For instance, some of the components would include the markdown converted to c++ and then the c++ converted to html. Both of these would require the class and queue stuff. The queue is still acting like a linked list as the node part contains the information and then the data is stored consecutively. \newline

\noindent While some of this we had known through the design stages, the questions that came up like this one were necessary to solidify what we already knew and make a final decision before beginning the implementation. \newline

\noindent In terms of schedule, based on the spike implemented last week in Assignment 5 last week it may take longer to implement the cpp to pdf part. This is because the pdf version is more complicated than converting to html. So, that portion may not be fully implemented by the end of this term. However, other than that there is not really any schedule changes as we had divided the large project of converting a markdown into a specific output into several components that build off of each other. Then for each of these components were divided among each of the team members. \newline
