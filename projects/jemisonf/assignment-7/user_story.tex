\section{User Stories}

\subsection{Specifying the File}

Worked on by: Sonica, Sean

Sean and Sonica collaborated remotely via Slack and Github to work on this project. They were also able to work together directly at team meetings.

This only took a few hours of work because it mainly involved piecing together existing files. It was also done concurrently with the "Specifying the Output Type" user story.

Currently, this user story is mostly complete. It works fine if the user enters "-o [filename].html" at the end of their command, but does not handle alternative argument orderings and fails silently if the user does not specify the output filename as .html.

Later work would involve revamping the command line interface to be more responsive to bad input.

\subsection{Specifying the Output Type}

Worked on by: Sonica, Sean

As in the previous user story, Sean and Sonica collaborated both remotely and in person at meetings.

This was also relatively quick, as the program only had to look for a specific flag and was not required to to interpret user input.

This user story currently interprets html just fine. However, it does not handle a -pdf or -stdout option, which was originally going to be part of the project. It also requires that the output type flag appear immediately after the input filename.

It would be nice to handle addtional file types in future versions, but that would require extensive additions to our backend code.

\subsection{Outputting HTML without CSS}

Worked on By: This part was worked on by Sean and Sonica. 

Like the other user stories, Sean and Sonica collaborated using various methods.

Problems: There was the issue of deciding which elements are going to be stored in MdData textnode text, or subnodes instead. We settled on leaving the syntax in the TextNode text field.

Time Length: This part took a few hours. 

Current Status: This part is still being implemented but headers and paragraphs are done, along with italics, bold, and strikethough in the testing stage. 

What is left: Depending on how much more of the Markdown spec we plan to implement, there is still work to be done on other syntax elements such as tables, horizontal rules, and images. 
